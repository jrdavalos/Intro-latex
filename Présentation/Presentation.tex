\documentclass[9pt]{beamer}

\usepackage[french]{babel}
\usepackage[utf8]{inputenc}%pour que latex comprenne les accents
\usepackage[T1]{fontenc}%pour que latex imprime les accents
\usepackage{multicol}%pour créer plusieurs colonnes

\usepackage{amsmath}
\usepackage{mathtools}
\usepackage{amsfonts}
\usepackage{amssymb}

\usetheme{CambridgeUS}%thème que l'on peut changer si l'on veut
\AtBeginSection[]{%pour mettre le plan au début de chaque section
	\begin{frame}{Table des matières}
		\tableofcontents[currentsection]
	\end{frame}
}

\title{\textbf{Mads Meier Jaeger \&  Richard Breen,\\ \og A Dynamic Model of Cultural Reproduction \fg{}}}
\author{Chaima Souli \& Julio Ricardo Davalos}
\institute{Master Quantifier en Sciences Sociales -- ENS/EHESS}
\date{Année 2020-2021}
\makeatletter
\setbeamercolor{autre}{fg = black, bg = white}
\setbeamercolor{title}{fg = red, bg = white}


\setbeamertemplate{title page}%pour changer la mise en forme de la page de titre
{
  \begin{centering}
    \begin{beamercolorbox}[sep=8pt,center, rounded = true, shadow = true]{title}%titre
      \usebeamerfont{title}\inserttitle\par%
      \ifx\insertsubtitle\@empty%
      \else%
        \vskip0.25em%
        {\usebeamerfont{subtitle}\usebeamercolor[fg]{subtitle}\insertsubtitle\par}%
      \fi%     
    \end{beamercolorbox}
    \begin{beamercolorbox}[sep=8pt,center,]{autre}%le reste
        \usebeamerfont{author}\insertauthor\\
        \usebeamerfont{institute}\insertinstitute\\
        \usebeamerfont{date}\insertdate
    \end{beamercolorbox}%\vskip0.5em
  \end{centering}
}

\begin{document}
	\begin{frame}
    \titlepage
	\end{frame}
	
\section*{Introduction}
\begin{frame}{Introduction}
    \begin{block}{Contexte}
        \begin{itemize}
            \item grand nombre de recherches sur la reproduction sociale à la suite de Bourdieu et Passeron
            \item[$\to$] pas de théorie vraiment unifiée mais tout converge vers un effet positif du capital culturel sur la réussite scolaire
            \item d'après les auteurs, manque de clarté dans les concepts bourdieusien
            \item[$\to$] en particulier sur le processus de transmission et de conversion du capital culturel ainsi que de la motivation qu'ont les parents à le faire
            \item pas/peu de modèles mathématiques sur la question
        \end{itemize}
    \end{block}
    \begin{alertblock}{Problématique}
	Comment compléter les zones d'ombre de la théorie bourdieusienne de la reproduction sociale ?
    \end{alertblock}
\end{frame}

\section*{Table des matières}
\begin{frame}{Table des matières}
	\tableofcontents[hideallsubsections]%pour l'annonce du plan
\end{frame}

\section{Modèles théoriques}
\subsection{Cadre d'analyse}
\begin{frame}{Cadre d'analyse}
	\begin{block}{Les auteurs filent la métaphore économique de Bourdieu concernant le capital culturel}
		\begin{itemize}
		\item[$\to$] on parle donc d'\og investissement \fg{} concernant la transmission
		\item[$\to$] les professeurs font des \og inputs \fg{} supplémentaires
		\end{itemize}
	\end{block}
	\begin{block}{Pour expliquer le choix d'investissement des parents: théorie du choix rationnel}
		\begin{itemize}
		\item[$\to$] les parents ajustent leur investissement de manière rationnelle
		\item[$\to$] ils maximisent une fonction d'utilité que l'on détaillera plus tard
		\item[$\to$] pour les auteurs cela permet d'approfondir et de dépasser le manque de clarté de la théorie bourdieusienne
		\end{itemize}
	\end{block}
\end{frame}
\subsection{Modèle statique}
\begin{frame}
	\begin{block}{Causalité entre capital culturel et position socio-économique}
		\begin{align}		
			\begin{dcases}
			Y_c=\rho_1E_c+\rho_2X_p+\rho_3A_c+Q_c\\
			E_c=\eta_1P_c + \eta_2X_p + \eta_3A_c+U_c \\
			P_c=\sigma_1C_c + \sigma_2A_c + J_c\\
			C_c=\beta_1\theta_p + \beta_2S_p + \beta_3X_p + \beta_4A_c + L_c
			\end{dcases}
		\end{align}
		\begin{itemize}
		\item $Y_c$ la position socio-économique de l'enfant une fois adulte;
		\item $X_p$ le capital social et économique des parents;
 		\item $A_c$ l'aptitude académique de l'enfant;
 		\item $E_c$ le niveau d'éducation final;
 		\item $P_c$ la performance scolaire;
 		\item $C_c$ le capital culturel de l'enfant;
 		\item $\theta_p$le capital culturel des parents investi dans l'éducation de l'enfant volontairement;
 		\item $S_p$ le stock total de capital culturel des parents;
 		\item $Q_c$, $U_c$, $J_c$ et $L_c$ la chance que peut avoir l'enfant à chaque étape.
		\end{itemize}
	\end{block}
\end{frame}

\subsection{Modèle dynamique}
\begin{frame}
	\begin{block}{Causalité entre capital culturel et performance scolaire en prenant le temps en compte}
		\begin{align}
		\begin{dcases}
		P_{ct}=\alpha_1T_t+\alpha_2A_c+\alpha_3X_{pt}+W_{ct}\\
		T_t = \phi_1P_{ct-1}+\phi_2C_{ct}+V_{ct}\\
		C_{ct}=\gamma_1C_{ct-1}+\gamma_2\theta_{pt}+\gamma_3S_p+\gamma_4X_{pt}+\gamma_5A_c+L_{ct}
		\end{dcases}
		\end{align}
		\begin{itemize}
		 \item $P_{ct}$ la performance scolaire à un moment $t$;
		 \item $T_t$ l'enseignement des professeurs en $t$
		\item $X_{pt}$ le capital social et économique des parents en $t$;
 		\item $A_c$ l'aptitude académique de l'enfant;
		\item $C_{ct}$ le capital culturel de l'enfant en $t$;
 		\item $\theta_{pt}$ le capital culturel des parents investi dans l'éducation d l'enfant volontairement;
 		\item $S_p$ le stock total de capital culturel des parents qui est investi involontairement;
 		\item $W_{ct}$, $V_{ct}$ et $L_{ct}$ la chance.\\
		\end{itemize}
		
		\begin{equation}
		\Rightarrow P_{ct} = m_0P_{ct-1}+m_1C_{ct-1}+m_2\theta_{pt}+m_3S_p+m_4A_c+m_5X_{pt}+\epsilon_{ct}
		\end{equation}
	\end{block}
\end{frame}
\begin{frame}
	\begin{alertblock}{Les parents sont considérés comme des maximisateurs}
	Ils maximisent:
		\begin{equation*}
		\mu P_{ct}-c(\theta_{pt})
		\end{equation*}
	C'est-à-dire les bénéfices d'un investissement dans leur enfant pondéré par leur altruisme et leur croyance dans l'effet d'une haute performance scolaire ($\mu$) auquel on soustrait le coût de l'investissement.
	\end{alertblock}
	\begin{block}{Hypothèses}
		\begin{itemize}
		\item les parents maximisent la fonction sous la contrainte de ce qu'ils pensent être les valeurs des $m_i$
		\item $\frac{\partial c(\theta_{pt})}{\partial\theta_{pt}}> 0$ (des investissements élevés sont plus coûteux que des investissements faibles)
		\item $\frac{\partial c(\theta_{pt})}{\partial S_{p}}< 0$ (plus le stock est grand, moins l'investissement est couteux relativement)
		\end{itemize}
	\end{block}
	\begin{block}{Résultats théoriques}
	\begin{equation*}
	\theta^*_{pt}=\frac{\mu \hat{m_2}_t}{h},\quad \text{avec}\quad \hat{m_2}_t=\hat{m_2}_{t-1}\left[1+\pi\left(\frac{P_{ct-1}-P_{ct-2}}{\theta_{t-1}-\theta_{t-2}}\right)\right],
	\end{equation*}
	$h$ le coût relatif de l'investissement et $\pi$ la sensibilité des parents aux résultats de leurs investissements
	\end{block}
\end{frame}

\section{Administration de la preuve}
\subsection{Données}
\begin{frame}{Données}
	\begin{block}{Panel utilisé}
		\begin{itemize}
		\item National Longitudinal Survey of Youth 1979 - Children and young adults
		\item[$\to$] données de panel bi-annuelles entre 1986 et 2010
		\item[$\to$] données sur des résultats scolaires et mesures de capital culturel de la mère et des enfants
		\item on ne prend que les enfants entre 10 et 14 ans
		\end{itemize}
	\end{block}
	\begin{block}{Variables utilisées}
		\begin{itemize}
		\item Capital culturel de l'enfant: indicateur de comportement de lecture entre 0 et 1;
		\item Capital culturel de la mère: investissement volontaire (entre 0 et 1) et involontaire (entre 0 et 1);
		\item Performance scolaire: test de lecture et test PIAT de mathématiques;
		\item Contrôles: échelles socio-économiques et démographiques.
		\end{itemize}
	\end{block}
\end{frame}

\subsection{Stratégie analytique}
\begin{frame}{Stratégie analytique}
	\begin{block}{Modèles dynamiques à tester}
	Pour l'évolution du capital culturel:
		\begin{equation}
		C_{i,t}=\tilde{\gamma}_1C_{i,t-1}+\tilde{\gamma}_2\theta_{i,t}+\tilde{\gamma}_3S_{i,t}+\tilde{\gamma}_4X_{i,t}+T+u_i+e_{1i,t}\label{C}
		\end{equation}
	Pour les résultats scolaires:
		\begin{equation}
		P_{i,t} = \tilde{m}_0P_{i,t-1}+\tilde{m}_1C_{i,t-1}+\tilde{m}_2\theta_{i,t}+\tilde{m}_3S_{i,t}+\tilde{m}_5X_{i,t}+T+u_i+e_{2i,t}\label{P}
		\end{equation}
	Pour l'investissement volontaire des parents:
		\begin{equation}
		\theta_{i,t}= \tilde{\tau}_1\theta_{i,t-1}+\tilde{\tau}_2\theta_{i,t-2}+\tilde{\tau}_3P_{i,t-1}+\tilde{\tau}_4(\theta_{i,t-2}\times P_{i,t-1})+\tilde{\tau}_5S_{i,t}+\tilde{\tau}_6X_{i,t}+T+u_i+e_{3i,t}\label{theta}
		\end{equation}
	\end{block}
	\begin{alertblock}{Remarques}
		\begin{itemize}
		\item les $u_i$ sont des effets fixes par individu, ils comprennent donc les $A_c$;
		\item les $T$ sont des effets fixes par année, on est donc dans des modèles \textit{within twoways};
		\item \eqref{theta} est un modèle simplifié du choix rationnel des parents vu plus haut
		\end{itemize}
	\end{alertblock}
\end{frame}

\subsection{Résultats}
\begin{frame}
	\begin{figure}[h]
		\begin{center}
		\includegraphics[width=\textwidth]{Tab1.png}
		\end{center}
	\end{figure}
\end{frame}

\begin{frame}
	\begin{figure}[h]
		\begin{center}
		\includegraphics[width=\textwidth]{Tab2.png}
		\end{center}
	\end{figure}
\end{frame}

\section*{Conclusion}
\begin{frame}{Conclusion}
	\begin{block}{Apports de l'article}
		\begin{itemize}
		\item Formalisation de la théorie bourdieusienne
		\item Les enfants accumulent bien du capital culturel transmis par les parents, ce qui a un impact positif sur leur scolarité
		\item Les parents ajustent leurs \og investissements \fg{} en fonction des résultats
		\end{itemize}
	\end{block}
	\begin{block}{Limites données par les auteurs}
		\begin{itemize}
		\item Les modèles font comme s'il n'y avait qu'un seul enfant;
		\item Les modèles ne prennent pas en compte l'hétérogénéité des \og retours sur investissement \fg{} possibles (par exemple en fonction de la classe sociale)
		\end{itemize}
	\end{block}
	\begin{alertblock}{Critiques}
		\begin{itemize}
		\item Assez peu de détails sur les variables de contrôle;
		\item Le capital culturel de l'enfant n'a qu'une dimension: la lecture $\to$ effet quasi mécanique de l'achat de livres par la famille;
		\item L'effet d'ajustement de l'investissement est significatif mais minuscule ($0,004$ dans les deux cas)
		\end{itemize}
	\end{alertblock}
\end{frame}
\end{document}